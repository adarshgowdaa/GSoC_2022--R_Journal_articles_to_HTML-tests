% !TeX root = RJwrapper.tex
\title{ggplot2: An R Package for to Create Elegant Data Visualisations
Using the Grammar of Graphics}
\author{by Mr.Abc Xyz and Mr.Qwe Mnb}

\maketitle

\abstract{%
A system for `declaratively' creating graphics, based on ``The Grammar
of Graphics''. You provide the data, tell `ggplot2' how to map variables
to aesthetics, what graphical primitives to use, and it takes care of
the details.
}

\hypertarget{introduction}{%
\section{Introduction}\label{introduction}}

Interactive data graphics provides plots that allow users to interact
them. One of the most basic types of interaction is through tooltips,
where users are provided additional information about elements in the
plot by moving the cursor over the plot.

This paper will first review some R packages on interactive graphics and
their tooltip implementations. A new package \CRANpkg{ggplot2} that
provides customized tooltips for plot, is introduced. Some example plots
will then be given to showcase how these tooltips help users to better
read the graphics.

\hypertarget{background}{%
\section{Background}\label{background}}

Some packages on interactive graphics include \CRANpkg{plotly}
\citep{plotly} that interfaces with Javascript for web-based interactive
graphics, \CRANpkg{crosstalk} \citep{crosstalk} that specializes
cross-linking elements across individual graphics. The recent R Journal
paper \CRANpkg{tsibbletalk} \citep{RJ-2021-050} provides a good example
of including interactive graphics into an article for the journal. It
has both a set of linked plots, and also an animated gif example,
illustrating linking between time series plots and feature summaries.

\hypertarget{customizing-tooltip-design-with}{%
\section{\texorpdfstring{Customizing tooltip design with
\pkg{ToOoOlTiPs}}{Customizing tooltip design with }}\label{customizing-tooltip-design-with}}

\pkg{ggplot2} is a packages for customizing tooltips in interactive
graphics, it features these possibilities.

\hypertarget{a-gallery-of-tooltips-examples}{%
\section{A gallery of tooltips
examples}\label{a-gallery-of-tooltips-examples}}

The \CRANpkg{palmerpenguins} data \citep{palmerpenguins} features three
penguin species which has a lovely illustration by Alison Horst in
Figure \ref{fig:penguins-alison}.

\begin{Schunk}
\begin{figure}
\includegraphics[width=1\linewidth,height=0.2\textheight]{penguins} \caption[Artwork by \@allison\_horst]{Artwork by \@allison\_horst}\label{fig:penguins-alison}
\end{figure}
\end{Schunk}

Table \ref{tab:penguins-tab-static} prints at the first few rows of the
\texttt{penguins} data:

\begin{Schunk}
\begin{table}

\caption{\label{tab:penguins-tab-static}A basic table}
\centering
\fontsize{7}{9}\selectfont
\begin{tabular}[t]{l|l|r|r|r|r|l|r}
\hline
species & island & bill\_length\_mm & bill\_depth\_mm & flipper\_length\_mm & body\_mass\_g & sex & year\\
\hline
Adelie & Torgersen & 39.1 & 18.7 & 181 & 3750 & male & 2007\\
\hline
Adelie & Torgersen & 39.5 & 17.4 & 186 & 3800 & female & 2007\\
\hline
Adelie & Torgersen & 40.3 & 18.0 & 195 & 3250 & female & 2007\\
\hline
Adelie & Torgersen & NA & NA & NA & NA & NA & 2007\\
\hline
Adelie & Torgersen & 36.7 & 19.3 & 193 & 3450 & female & 2007\\
\hline
Adelie & Torgersen & 39.3 & 20.6 & 190 & 3650 & male & 2007\\
\hline
\end{tabular}
\end{table}

\end{Schunk}

Figure \ref{fig:penguins-ggplot} shows an plot of the penguins data,
made using the \CRANpkg{ggplot2} package.

\begin{Schunk}
\begin{Sinput}
penguins %>% 
  ggplot(aes(x = bill_depth_mm, y = bill_length_mm, 
             color = species)) + 
  geom_point()
\end{Sinput}
\begin{figure}
\includegraphics{article_files/figure-latex/penguins-ggplot-1} \caption[A basic non-interactive plot made with the ggplot2 package on palmer penguin data]{A basic non-interactive plot made with the ggplot2 package on palmer penguin data. Three species of penguins are plotted with bill depth on the x-axis and bill length on the y-axis. Visit the online article to access the interactive version made with the plotly package.}\label{fig:penguins-ggplot}
\end{figure}
\end{Schunk}

\hypertarget{summary}{%
\section{Summary}\label{summary}}

We have displayed various tooltips that are available in the package
\pkg{ggplot2}.

\bibliography{RJreferences.bib}

\address{%
Mr.Abc Xyz\\
University of R\\%
Department of R Journal\\ Open source, R Community\\
%
\url{https://www.r-project.org/}\\%
\textit{ORCiD: \href{https://orcid.org/0000-1721-1511-1101}{0000-1721-1511-1101}}\\%
\href{mailto:abc.xyz@r-project.org}{\nolinkurl{abc.xyz@r-project.org}}%
}

\address{%
Mr.Qwe Mnb\\
University of R\\%
Department of R Journal\\ Open source, R Community\\
%
\url{https://www.r-project.org/}\\%
\textit{ORCiD: \href{https://orcid.org/0000-0002-0912-0225}{0000-0002-0912-0225}}\\%
\href{mailto:qwe.mnb@r-project.org}{\nolinkurl{qwe.mnb@r-project.org}}%
}
